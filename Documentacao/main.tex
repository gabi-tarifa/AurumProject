%abtex2modeloartigo.tx,v-1.9.7laurocesr
%% Copyright 2012-2018 by abnTeX2 group at http://www.abntex.net.br/ 
%%
%% This work may be distributed and/or modified under the
%% conditions of the LaTeX Project Public License, either version 1.3
%% of this license or (at your option) any later version.
%% The latest version of this license is in
%%   http://www.latex-project.org/lppl.txt
%% and version 1.3 or later is part of all distributions of LaTeX
%% version 2005/12/01 or later.
%%
%% This work has the LPPL maintenance status `maintained'.
%% 
%% The Current Maintainer of this work is the abnTeX2 team, led
%% by Lauro César Araujo. Further information are available on 
%% http://www.abntex.net.br/
%%
% ------------------------------------------------------------------------
% ------------------------------------------------------------------------
% abnTeX2: Modelo de Artigo Acadêmico em conformidade com
% ABNT NBR 6022:2018: Informação e documentação - Artigo em publicação 
% periódica científica - Apresentação
% ------------------------------------------------------------------------
% ------------------------------------------------------------------------

\documentclass[
	% -- opções da classe memoir --
	article,			% indica que é um artigo acadêmico
	12pt,				% tamanho da fonte
	oneside,			% para impressão apenas no recto. Oposto a twoside
	a4paper,			% tamanho do papel. 
	% -- opções da classe abntex2 --
	%chapter=TITLE,		% títulos de capítulos convertidos em letras maiúsculas
	%section=TITLE,		% títulos de seções convertidos em letras maiúsculas
	%subsection=TITLE,	% títulos de subseções convertidos em letras maiúsculas
	%subsubsection=TITLE % títulos de subsubseções convertidos em letras maiúsculas
	% -- opções do pacote babel --
	english,			% idioma adicional para hifenização
	brazil,				% o último idioma é o principal do documento
	sumario=tradicional
	]{abntex2}


% ---
% PACOTES
% ---

% ---
% Pacotes fundamentais 
% ---
\usepackage{sbc-template}
\usepackage{times}			% Usa a fonte Times new Roman
\usepackage[T1]{fontenc}		% Selecao de codigos de fonte.
\usepackage[utf8]{inputenc}		% Codificacao do documento (conversão automática dos acentos)
\usepackage{indentfirst}		% Indenta o primeiro parágrafo de cada seção.
\usepackage{nomencl} 			% Lista de simbolos
\usepackage{color}				% Controle das cores
\usepackage{graphicx}			% Inclusão de gráficos
\usepackage{microtype} 			% para melhorias de justificação
\usepackage{multirow}
\usepackage{adjustbox}
\usepackage{float}
\usepackage{titlesec}
\usepackage{tikz}               % para ticks
\usepackage{longtable}          % para quebra de tabela por página
\usepackage{tabularray}         % formatação da coluna e linha da tabela

% Deixa o \subsubsection igual ao \subsection: negrito, tamanho igual, sem itálico, com espaçamento igual
\titleformat{\subsubsection}[hang]{\normalfont\bfseries}{\thesubsubsection}{1em}{}

% Ajusta espaçamento antes e depois do \subsubsection
\titlespacing*{\subsubsection}{0pt}{1.5ex plus .2ex minus .2ex}{1ex plus .2ex}
% ---
		
% ---
% Pacotes adicionais, usados apenas no âmbito do Modelo Canônico do abnteX2
% ---
\usepackage{lipsum}				% para geração de dummy text
% ---
		
% ---
% Pacotes de citações
% ---
\usepackage[brazilian,hyperpageref]{backref}	 % Paginas com as citações na bibl
\usepackage[alf]{abntex2cite}	% Citações padrão ABNT
% ---

% ---
% Configurações do pacote backref
% Usado sem a opção hyperpageref de backref
\renewcommand{\backrefpagesname}{Citado na(s) página(s):~}
% Texto padrão antes do número das páginas
\renewcommand{\backref}{}
% Define os textos da citação
\renewcommand*{\backrefalt}[4]{
	\ifcase #1 %
		Nenhuma citação no texto.%
	\or
		Citado na página #2.%
	\else
		%Citado #1 vezes nas páginas #2.%
        Citado nas páginas #2.%
	\fi}%
% ---

\def\checkmark{\tikz\fill[scale=0.4](0,.35) -- (.25,0) -- (1,.7) -- (.25,.15) -- cycle;} 

% --- Informações de dados para CAPA e FOLHA DE ROSTO ---

\title{Documento de arquitetura de software de Aurum}

\author{Arthur Silva Vieira\inst{1}, Francine Midori Nakasuga Oda\inst{1}, Gabriel Tarifa de Oliveira\inst{1},\\ Gustavo Batista Ramos\inst{1}, Thomas Stephan Carmo\inst{1},   Rafael Oliveira Muñoz\inst{1} }

\address{Instituto Federal de Educação, Ciência e Tecnologia de São Paulo - Câmpus São Paulo
}


% ---

% ---
% Configurações de aparência do PDF final

% alterando o aspecto da cor azul
\definecolor{blue}{RGB}{41,5,195}

% informações do PDF
\makeatletter
\hypersetup{
     	%pagebackref=true,
		pdftitle={\@title}, 
		pdfauthor={\@author},
    	pdfsubject={Modelo de artigo científico com abnTeX2},
	    pdfcreator={LaTeX with abnTeX2},
		pdfkeywords={abnt}{latex}{abntex}{abntex2}{atigo científico}, 
		colorlinks=true,       		% false: boxed links; true: colored links
    	linkcolor=blue,          	% color of internal links
    	citecolor=blue,        		% color of links to bibliography
    	filecolor=magenta,      		% color of file links
		urlcolor=blue,
		bookmarksdepth=4
}
\makeatother
% --- 

% ---
% compila o indice
% ---
\makeindex
% ---

% ---
% Altera as margens padrões
% ---
\setlrmarginsandblock{3cm}{3cm}{*} %{ESQUERDA}{DIREITA}
\setulmarginsandblock{3.5cm}{2.5cm}{*} %{SUPERIOR}{INFERIOR}
\checkandfixthelayout
% ---

% --- 
% Espaçamentos entre linhas e parágrafos 
% --- 

% O tamanho do parágrafo é dado por (está definido dentro do sbc-template.sty):
%\setlength{\parindent}{1.3cm}

% Controle do espaçamento entre um parágrafo e outro (está definido dentro do sbc-template.sty):
%\setlength{\parskip}{0.2cm}  % tente também \onelineskip

% Espaçamento simples
\SingleSpacing


% ----
% Início do documento
% ----
\begin{document}

% Seleciona o idioma do documento (conforme pacotes do babel)
%\selectlanguage{english}
\selectlanguage{brazil}


% Retira espaço extra obsoleto entre as frases.
\frenchspacing 

% ----------------------------------------------------------
% ELEMENTOS PRÉ-TEXTUAIS
% ----------------------------------------------------------

%---
%
% Se desejar escrever o artigo em duas colunas, descomente a linha abaixo
% e a linha com o texto ``FIM DE ARTIGO EM DUAS COLUNAS''.
% \twocolumn[    		% INICIO DE ARTIGO EM DUAS COLUNAS
%
%---

% página de titulo principal (obrigatório)
\maketitle

% titulo em outro idioma (opcional)

% inserir ABSTRACT e RESUMO aqui


% ]  				% FIM DE ARTIGO EM DUAS COLUNAS
% ---

%\begin{center}\smaller
%\textbf{Data de submissão e aprovação}: elemento c    obrigatório. Indicar dia, mês e ano

%\textbf{Identificação e disponibilidade}: elemento opcional. Pode ser indicado o endereço eletrônico, DOI, suportes e outras informações relativas ao acesso.
%\end{center}

% ----------------------------------------------------------
% ELEMENTOS TEXTUAIS
% ----------------------------------------------------------
\textual

% ----------------------------------------------------------
% Introdução
% ----------------------------------------------------------
\section{Introdução}
    %O projeto visa abranger os temas entretenimento e cultura, com seu foco em primeira instância os cinemas. https://wiki.sj.ifsc.edu.br/index.php/Dicas_de_LaTeX
    
 A educação financeira é um direito fundamental do cidadão, desempenhando papel crucial no desenvolvimento social e pessoal. Contudo, apesar de sua importância, grande parte da população brasileira ainda enfrenta barreiras significativas para acessar informações adequadas sobre finanças. Esses obstáculos incluem a escassez de recursos para investir em cursos especializados, a falta de incentivo para aprofundar o tema desde o ensino fundamental e a dificuldade em compreender conceitos financeiros devido à linguagem técnica e complexa.

Nesse cenário, segundo dados do Instituto Brasileiro de Geografia e Estatística (\citeonline{ibge2024}), embora a taxa de pobreza tenha diminuído significativamente, caindo de 67,7 milhões para 59 milhões de pessoas, ainda é preocupante o número de cidadãos que permanecem à margem da sociedade. Isto reforça a urgência de iniciativas que promovam a educação financeira como uma ferramenta de inclusão social e empoderamento.

Como consequência da falta de conhecimento financeiro, algo que pode afetar a vida das pessoas é o uso do cartão de crédito. De acordo com \citeonline{pires2008planejamento}, muitos brasileiros perdem o controle dos gastos com o cartão devido à falta de educação sobre os custos reais envolvidos e a ausência de estratégias eficazes de gestão financeira. O uso excessivo do crédito e a escolha por parcelamentos com as menores parcelas possíveis frequentemente resultam em uma dívida muito maior do que o valor inicialmente contratado, agravando ainda mais a situação financeira do consumidor.

Neste contexto, mitos alunos da Educação para Jovens e Adultos (EJA) são trabalhadores que gerenciam diariamente suas finanças, e a educação financeira auxilia no desenvolvimento de uma cultura de prevenção e proteção, promovendo a cidadania e a melhoria da qualidade de vida, segundo \citeonline{hurtado2020importancia}. Portanto, a educação financeira desde a base deve ser prioridade.

Assim, surge a necessidade de ferramentas educativas inovadoras e acessíveis. A proposta desta aplicação web é desenvolver uma plataforma de aprendizagem interativa e gamificada, que permita aos usuários aprender e praticar conceitos financeiros de forma progressiva e adaptativa, de acordo com seu nível de conhecimento e ritmo.

Dessa forma, por meio de módulos interativos e desafios diários, a plataforma busca capacitar os usuários para que tomem decisões financeiras mais conscientes, como controlar gastos, entender as nuances dos investimentos, negociar dívidas e planejar para o futuro. Além disso, a utilização de tecnologia de aprendizado adaptativo permite que o conteúdo seja personalizado, atendendo às necessidades específicas de cada usuário, desde os mais iniciantes até aqueles que já possuem algum conhecimento prévio.

Em suma, a missão deste projeto é expandir o acesso à educação financeira, oferecendo uma ferramenta simples, envolvente e eficaz, capaz de transformar a realidade financeira de milhares de pessoas. Ao buscar isso, utilizamos uma linguagem acessível e evitamos jargões técnicos que costumam afastar o público geral. O objetivo final é empoderar os cidadãos, oferecendo-lhes as ferramentas necessárias para melhorar sua qualidade de vida, reduzir as desigualdades sociais e promover um crescimento econômico mais sustentável para todos.

\subsection{Objetivos}
    Levando em consideração a necessidade de detalhes da aplicação, os objetivos foram divididos em: geral e específico.
\subsubsection{Objetivo Geral}
    O objetivo do projeto é criar uma plataforma que auxilie pessoas de todas as idades no estudo de educação financeira, fornecendo informações de forma simples para fácil entendimento, seguido de testes para treinar e avaliar seu aprendizado.
\subsubsection{Objetivos Específicos}
    Para atingir o objetivo geral do projeto foram definidos os seguintes objetivos específicos:
        \begin{enumerate}[label=\alph*)]
            \item Facilitar o acesso à educação financeira, de forma lúdica, por meio de aplicação web.
            \item Incentivar o planejamento financeiro
            \item Apresentar princípios financeiros básicos 
        \end{enumerate}
       
\section{Justificativa}    
    A educação financeira é um elemento essencial para a construção de uma sociedade mais equilibrada e justa. Entretanto, no Brasil, muitas pessoas ainda enfrentam dificuldades para lidar com o próprio dinheiro, seja por falta de conhecimento, seja por influências do consumo excessivo. De acordo com o Instituto Brasileiro de Geografia e Estatística (IBGE), mesmo com a redução do índice de pobreza nos últimos anos, milhões de brasileiros ainda vivem em situação de vulnerabilidade econômica, muitas vezes agravada pelo endividamento descontrolado.

    Foi realizada uma pesquisa com 217 participantes, com o objetivo de compreender melhor o perfil e as dificuldades enfrentadas pelo público em relação à educação financeira. Os resultados do questionário aplicado reforçam a importância da criação de uma plataforma acessível, interativa e \textit{gamificada} sobre finanças.

    Um dos dados mais reveladores foi a resposta à pergunta "Como você descreveria seus conhecimentos em educação financeira?", em que 51,6\% dos participantes declarou possuir conhecimentos básicos e 12\% responderam que não possuem nenhum conhecimento. Isso demonstra que, embora o tema esteja ganhando mais espaço nos debates sociais, ainda há uma lacuna significativa na formação financeira da população.

    Outro aspecto relevante: "tenho dificulade de controlar meus gastos". O que revela um ponto crítico: muitas pessoas desconhecem métodos de organização financeira e tomada de decisão, que é o básico. Isso limita o potencial de crescimento financeiro pessoal e mostra a urgência de métodos que desmistifiquem e tornem acessível esse tipo de conteúdo. A proposta desta plataforma contempla esse aspecto ao oferecer módulos específicos sobre controle de gastos, com explicações simples e interativas.

    Em seguida, as seguintes maiores dificuldades relatadas foram não conseguir economizar, que está diretamente ligado com a primeira maior dificuldade, e não compreender sobre como funcionam investimentos, o que reforça a necessidade de ensinar estratégias práticas de educação financeira. Essa combinação entre desconhecimento técnico (com os investimentos), e dificuldade na gestão cotidiana evidencia o quanto a população precisa de uma solução completa, que ensine tanto o básico quanto o avançado.

    Além disso, constatou-se que a concentração da faixa etária entre 35 e 54 anos. Esse público está em uma fase da vida onde já existe a preocupação com aposentadoria, mas muitas vezes sem ter recebido qualquer educação formal sobre finanças. Para esse grupo, uma plataforma online, moderna e compatível com os hábitos digitais representa uma alternativa prática e eficaz.

    Bem como revelou que a maioria das pessoas raramente faz um planejamento financeiro, evidenciando uma falha estrutural no desenvolvimento de hábitos financeiros saudáveis. Por isso, a plataforma propõe desafios diários e objetivos de longo prazo, com o intuito de incentivar o planejamento constante e o acompanhamento do progresso do usuário.

    Quando questionados sobre o formato de aprendizado mais interessante, o público demonstrou forte preferência por formas dinâmicas e interativas, como simulações práticas e conteúdo visual. Isso valida o uso da gamificação como uma metodologia principal do projeto, com o objetivo de tornar o aprendizado mais leve, envolvente e contínuo.

    Por fim, o dado que mais reforça a relevância do projeto: a grande maioria dos participantes declarou que se interessaria por uma plataforma interativa e \textit{gamificada} para aprender sobre finanças. Isso confirma que a proposta não só responde a uma necessidade real, como também desperta o interesse do público, aumentando o engajamento e o potencial de transformação individual e social.

    Embora existam jogos que ensinem sobre educação financeira, o Aurum, além de possuir módulos interativos e desafios diários, apresenta uma abordagem inovadora que combina interatividade, \textit{gamificação} e aprendizado adaptativo. E demonstra seu diferencial por personalizar o conteúdo de acordo com as necessidades específicas de cada usuário, desde iniciantes até aqueles com conhecimento prévio, os usuários são capacitados a tomar decisões financeiras mais conscientes, como controlar gastos, compreender investimentos, negociar dívidas e planejar o futuro.
    
    Ao expandir o acesso à educação financeira, esse projeto tem o potencial de ajudar milhares de brasileiros a adquirirem mais autonomia sobre suas finanças, evitando armadilhas do consumo impulsivo e do crédito mal utilizado. Dessa maneira, além de beneficiar cada indivíduo, a iniciativa contribui para o desenvolvimento de uma sociedade mais consciente e preparada para enfrentar desafios econômicos.

    Portanto, a implementação dessa plataforma se mostra uma solução inovadora e necessária, pois busca preencher uma lacuna existente no ensino tradicional e promover um futuro financeiro mais seguro para todos.

\section{Referencial Teórico}
\subsection{A Educação Financeira no Brasil e suas Implicações para o Desenvolvimento de Soluções Digitais}
    A educação financeira é muito mais do que aprender a guardar dinheiro ou fazer contas. Ela é essencial para que as pessoas consigam organizar suas vidas, tomar decisões conscientes sobre o uso do dinheiro e melhorar sua qualidade de vida. No Brasil, ela tem se tornado cada vez mais importante, especialmente depois de mudanças na economia, como a queda da inflação e o aumento do acesso ao crédito.

Antigamente, a educação financeira era voltada só para quem tinha dinheiro para investir, ou seja, era algo restrito às pessoas mais ricas. A maioria da população enfrentava problemas como inflação alta, pouco acesso a bancos e crédito difícil. Até os anos 1990, quase não se falava sobre isso nas escolas, e os próprios Parâmetros Curriculares Nacionais (PCN) não mencionavam diretamente a educação financeira  \citeonline{giordano2019educaccao}.

Mas isso começou a mudar a partir dos anos 2000, principalmente com a criação do Plano Real, que controlou a inflação, e o surgimento da nova classe média. Mais gente passou a consumir, mas sem estar preparada financeiramente. Resultado: muita gente se endividou. Por isso, ficou claro que era preciso ensinar as pessoas a planejar melhor suas finanças. Em 2010, foi criada a Estratégia Nacional de Educação Financeira (ENEF), com o objetivo de tornar os brasileiros mais conscientes e preparados para lidar com o dinheiro.

Com o tempo, a educação financeira foi se popularizando. Livros como Pai Rico, Pai Pobre e Casais Inteligentes Enriquecem Juntos começaram a mostrar que qualquer pessoa pode aprender a cuidar do seu dinheiro. Essa ideia vai ao encontro do que dizem pesquisadores como Silva e Powell (2013), que defendem uma educação financeira crítica, que ajude o aluno a entender a realidade em que vive e a tomar decisões responsáveis.

Outro fator importante foi a entrada da educação financeira na Base Nacional Comum Curricular (BNCC). A BNCC trouxe a educação financeira para dentro das escolas, desde o ensino fundamental até o ensino médio. Ela propõe que o tema seja trabalhado em várias disciplinas, como Matemática, Geografia, Ciências e até Língua Portuguesa, e de forma conectada com o dia a dia dos alunos. Um exemplo é quando aprendemos sobre porcentagem e juros e usamos isso para entender as faturas do cartão de crédito, boletos e até o impacto das contas de energia no orçamento familiar  \citeonline{giordano2019educaccao}.

A BNCC também valoriza o uso da tecnologia para ajudar nesse processo. Ela sugere o uso de planilhas, aplicativos, jogos digitais e simuladores de juros compostos para ajudar os alunos a entenderem melhor como o dinheiro funciona e como se organizar financeiramente  \citeonline{giordano2019educaccao}.

Hoje, com o avanço da tecnologia e o fácil acesso à internet, surgiram novas formas de aprender sobre finanças: jogos educativos, plataformas gamificadas e até a inteligência artificial estão sendo usadas para ensinar de forma divertida e interativa. Segundo a OECD, ensinar educação financeira desde cedo é uma das formas mais justas e eficazes de alcançar toda uma geração \citeonline{giordano2019educaccao}.

Quando as pessoas entendem melhor como lidar com o dinheiro, elas tomam decisões mais equilibradas, evitam dívidas e ajudam a construir uma sociedade mais justa. Como dizem os autores do artigo, a educação financeira deve ser crítica, contextualizada e ligada à realidade dos alunos. Mais do que saber fazer contas, ela ajuda a desenvolver o pensamento crítico e a preparar os jovens para os desafios da vida  \citeonline{giordano2019educaccao}.

Por isso, investir em educação financeira nas escolas é investir no futuro de todos nós. Uma população mais consciente financeiramente significa menos inadimplência, mais bem-estar social e uma economia mais forte e equilibrada.
    
\subsection{Construção da Educação Financeira no Brasil: Fases, Aprendizados e Aplicações Tecnológicas}
    A educação financeira tem se tornado um tema central nas políticas públicas brasileiras, especialmente no contexto escolar. Definida como o processo pelo qual indivíduos aprimoram sua compreensão sobre produtos financeiros para tomar decisões informadas e responsáveis, a educação financeira se fortaleceu como ferramenta para o bem-estar econômico e social.

    No desenvolvimento de sistemas tecnológicos voltados à gestão financeira, compreender o cenário institucional e político da educação financeira no Brasil é essencial para que tais ferramentas se alinhem às diretrizes públicas e tenham maior eficácia.

    A institucionalização da educação financeira no país está ligada a uma agenda internacional de inclusão financeira. Segundo \citeonline{cunha2020mercado}, organismos como a OCDE e a INFE influenciaram sua implementação, promovendo estratégias voltadas ao letramento financeiro desde o início dos anos 2000.

    No Brasil, a Estratégia Nacional de Educação Financeira (ENEF), formalizada pelo Decreto nº 7.397/2010, articula instituições públicas e privadas, consolidando uma simbiose entre o setor financeiro e políticas governamentais. A atuação da ENEF ocorre por meio de projetos nas escolas, sendo a Associação de Educação Financeira do Brasil (AEF) a principal responsável pela implementação.

    A OCDE desempenha um papel central na formulação das diretrizes globais de educação financeira. Segundo \citeonline{cunha2020mercado}, após a crise de 2008, a organização passou a incentivar a educação financeira como prioridade, principalmente entre crianças e jovens, motivando o Brasil a alinhar sua estratégia aos parâmetros internacionais.

    Embora a expansão da educação financeira pareça positiva, \citeonline{cunha2020mercado} faz uma análise crítica da política pública adotada no Brasil. Para a autora, esse modelo reflete um projeto de sociedade não discutido democraticamente, consolidando um ideal de sujeito consumidor adaptado ao sistema financeiro vigente.

    A lógica pedagógica estaria sendo substituída por uma lógica mercadológica, esvaziando o potencial emancipador da educação. A pouca participação dos atores sociais interessados reforça essa crítica, levantando questionamentos sobre os impactos reais da educação financeira escolar.

    O papel dos sistemas digitais nesse contexto deve ser analisado com cautela. Ferramentas tecnológicas de gestão financeira precisam ser desenvolvidas considerando o cenário sociopolítico e seu impacto sobre os usuários. Em vez de apenas reforçar a lógica de consumo responsável, podem atuar na formação de sujeitos críticos e autônomos.

    Isso implica transformar as soluções tecnológicas em plataformas de conscientização, promovendo diálogos e personalização do ensino financeiro. A consolidação da educação financeira como política pública no Brasil reflete a expansão do mercado financeiro para novos públicos, especialmente os mais vulneráveis.

    Segundo \citeonline{cunha2020mercado}, a política atual articula instituições nacionais e internacionais sem um diálogo significativo com a comunidade escolar, resultando na inserção de conteúdos técnicos sem participação democrática.

\subsection{Abordagens, Evolução e Impacto no Ensino Escolar}
    A educação financeira desempenha um papel essencial na formação de cidadãos conscientes e autônomos, sendo um tema cada vez mais presente no contexto escolar. De acordo com \citeonline{cordeiro2019uso}, sua relevância tem sido fortalecida por iniciativas como a Estratégia Nacional de Educação Financeira (ENEF), que busca estruturar e disseminar práticas de ensino voltadas à gestão financeira pessoal. No Brasil, esse movimento ganhou força com a inserção da temática na Base Nacional Comum Curricular (BNCC), promovendo habilidades financeiras desde o ciclo de alfabetização.

    Considerando o impacto da tecnologia no ensino, softwares de educação financeira podem atuar como facilitadores na construção do conhecimento sobre planejamento econômico. Segundo \citeonline{cordeiro2019uso}, as crianças estão expostas ao consumo desde cedo, sendo influenciadas por propagandas e estímulos mercadológicos, o que reforça a necessidade de uma abordagem pedagógica direcionada. A integração de ferramentas tecnológicas nesse processo pode ampliar a compreensão de conceitos financeiros, permitindo que os alunos vivenciem práticas simuladas de gestão de recursos.

    O uso de metodologias interativas tem sido uma tendência na educação financeira, e \citeonline{cordeiro2019uso} destaca as Histórias em Quadrinhos (HQs) como um recurso eficaz para engajar alunos no aprendizado. As HQs podem ser incorporadas a softwares educacionais para ilustrar situações cotidianas, como orçamento, consumo consciente e poupança, tornando a aprendizagem mais acessível e envolvente. Além disso, segundo os autores, a representação gráfica estimula o pensamento crítico e promove a contextualização do ensino financeiro dentro da realidade das crianças.

    Outro aspecto relevante para o desenvolvimento de um software de educação financeira é sua adequação às diretrizes da BNCC. Conforme \citeonline{cordeiro2019uso}, essa estrutura curricular enfatiza temas como o sistema monetário brasileiro, o reconhecimento de cédulas e moedas e a equivalência de valores, fundamentais para a construção de uma base sólida no aprendizado financeiro. Um software pode potencializar esse aprendizado por meio de desafios interativos, simuladores de transações e ferramentas de acompanhamento do progresso do usuário.

    Por fim, ressalta-se que a educação financeira não deve apenas instruir sobre consumo responsável, mas também incentivar a autonomia crítica dos estudantes em relação ao sistema econômico. Softwares educacionais podem ser projetados para estimular essa consciência, promovendo reflexões sobre decisões financeiras e incentivando o planejamento sustentável. Dessa forma, integrar conceitos teóricos a soluções digitais pode contribuir significativamente para a formação de indivíduos mais preparados para lidar com desafios financeiros ao longo da vida.

    Além da abordagem tradicional, novas metodologias tecnológicas, como plataformas gamificadas, têm se mostrado eficientes na aprendizagem financeira. Segundo \citeonline{cordeiro2019uso}, a inclusão de elementos interativos e desafios práticos pode aumentar o engajamento dos estudantes, tornando a experiência de aprendizagem mais dinâmica e eficaz.

\section{Materiais e Métodos}
Para o desenvolvimento da plataforma Aurum, foram utilizadas ferramentas tecnológicas modernas, com o objetivo de garantir uma estrutura escalável e de fácil manutenção. Os materiais empregados abrangem desde linguagens de programação e banco de dados até ambientes de modelagem e controle de versões. No que diz respeito aos métodos, adotou-se uma abordagem com foco em prototipação rápida e validação contínua, permitindo ajustes ágeis durante todo o processo de construção do sistema.
\subsection{Equipe}

    \caption{\label{quadro_integrantes}Integrantes da equipe}\\
        \begin{tabular}{|c|c|c|c|c|c|c|}
        \hline
            Papéis         & Arthur                    & Francine                  & Gabriel                  & Gustavo                  & Rafael                     & Thomas                    \\ \hline
            Back-end       &  \checkmark                         &                           &      \checkmark           &                           &                 &                 \\ \hline
            Front-end      &   \checkmark              &                 &                           &                 &                           &            \checkmark               \\ \hline
            Banco de Dados &      \checkmark                     &                           & \checkmark                &                           &                 &                           \\ \hline
	    Testes 	   &			       & 		\checkmark		   & 	       & 			   & 		\checkmark	       &		\checkmark	   \\ \hline
            Documentação   &                 & \checkmark                &                 & \checkmark                &                 &                 \\ \hline
            Design         &                           &                           &                &                 &                           &                 \checkmark          \\ \hline
            Gestão         &                           &          \checkmark                  &                &                           &                 &              \checkmark             \\ \hline
        \end{tabular}
    \fonte{Autores.}

\subsection{Materiais}
O desenvolvimento da plataforma Aurum contou com o uso de diversas ferramentas tecnológicas para garantir eficiência e escalabilidade. A seguir, listam-se as principais:

Banco de Dados: MySQL, utilizado para armazenamento relacional dos dados da plataforma, com suporte a gatilhos e transações.

Linguagens: HTML5 e CSS3 para o front-end, com suporte a JavaScript para interações básicas. Python e Flask para back-end.

Modelagem: A modelagem de dados foi realizada com base no modelo Entidade-Relacionamento (MER), elaborado por meio da ferramenta Draw.io.

Editor de Código: Visual Studio Code.

Versionamento: GitHub foi utilizado para controle de versões e colaboração.


\subsection{Métodos}
Foi adotada uma abordagem incremental no desenvolvimento do sistema, com foco na prototipação rápida e validação contínua dos requisitos. A construção do projeto foi dividida nas seguintes etapas:

Levantamento de Requisitos Funcionais e Não Funcionais;

Modelagem de Dados utilizando o modelo relacional;

Implementação das estruturas de banco de dados via SQL;

Criação da interface web inicial (HTML + CSS);

Integração das funcionalidades básicas de login, cadastro, e visualização de tarefas e módulos;

Implantação de lógica de premiação por meio de procedimentos SQL, gatilhos e consultas temporais.


\section{Desenvolvimento}

Esta seção é destinada às definiçôes do sistema, àquilo que deve ser atentido, seja funcionamento, seja desempenho do algoritmo. 

\subsection{Requisitos}
Serão apresentados os requisitos que dierecionam o desenvolvimento do sistema. Os requisitos foram divididos em duas categorias: funcionais, que descrevem os comportamentos e funcionalidades esperadas do sistema; e não funcionais, que estabelecem restrições e qualidades que o sistema deve possuir. Esses requisitos foram definidos com base nos objetivos do projeto e nas necessidades dos usuários finais.


\subsubsection{Requisitos Funcionais}

\begin{center}
\begin{tabular}{|c|l|p{6cm}|}
\hline
\textbf{Código} & \textbf{Requisito} & \textbf{Descrição} \\
\hline
RF01 & Cadastrar usuário & O usuário deve ser capaz de se cadastrar na plataforma.\\
\hline
RF02 & Teste de nível & O usuário deve fazer um teste de até 12 questões para recomendar as tarefas de acordo com o nível.\\
\hline
RF03 & Disponibilização de tarefas & As tarefas devem ser disponibilizadas para o usuário de acordo com sua aptidão. \\
\hline
RF04 & Sistema de pontuação & O sistema deve contabilizar os pontos do usuário em cada tarefa realizada de acordo com o desempenho. \\
\hline
RF05 & Premiações por mérito & O sistema deve distribuir premiações por dias de conclusão de tarefas consecutivos, tarefas feitas com perfeição, entre outros. \\
\hline
RF06 & Criação de grupos & O sistema deve disponibilizar a criação de grupos pelo usuário, adicionando amigos de sua lista ao grupo. \\
\hline
RF07 & Adição de amigos & O usuário deve ser capaz de enviar pedidos de amizade para outros usuários. \\
\hline
RF08 & Compras na loja & As compras devem ser realizadas utilizando as "Aurum Coins", moedas do sistema. As compras não são possíveis com dinheiro real. \\
\hline
\end{tabular}
\end{center}


\subsubsection{Requisitos Não Funcionais}

\begin{center}
\vspace{-0.0cm}

\begin{tabular}{|c|l|p{8cm}|}
\hline
\textbf{Código} & \textbf{Requisito} & \textbf{Descrição} \\
\hline
RNF01 & Disponibilidade & O servidor deve estar disponível 24 horas por dia, 7 dias por semana, com tolerância de 0,5\% de falhas. \\
\hline
RNF02 & Desempenho & O servidor deve responder em, no máximo, 0,5 segundos a todas as requisições recebidas. \\
\hline
RNF03 & Compatibilidade & Deve ser compatível com os principais sistemas operacionais (Windows, Linux). \\
\hline
RNF04 & Otimização & O tempo de resposta para atualização da contagem de pontos da aplicação deve ser de no máximo 3 segundos. \\
\hline
RNF05 & Acessibilidade & O sistema deve ser acessível para Usuários que utilizam dispositivos adaptados (Alto Contraste, Redução de Movimento e Daltonismo). \\
\hline
\end{tabular}
\end{center}

\subsection{Regras de Negócio}

\begin{center}
\vspace{-0.0cm}

\begin{tabular}{|c|l|p{8cm}|}
\hline
\textbf{Código} & \textbf{Regra de Negócio} & \textbf{Descrição} \\
\hline
RN01 & Senhas & A senha do usuário deverá conter de 6 a 16 dígitos, contendo pelo menos uma letra maiúscula, uma letra minúscula, um carácter especial e um número. \\
\hline
RM02 & Cadastro & Usuário só poderá acessar o sistema se ele possuir um cadastro; \\
\hline
RN03 & Módulos de Ensino & Os módulos são liberados de forma sequencial. Cada módulo inclui teoria simples, exemplos práticos e mini-testes. O usuário só avança de nível se obtiver um desempenho mínimo de 70\% nos testes. \\
\hline
RN04 & Restrição etária & Menores de 12 não devem acessar a plataforma; \\
\hline
RN05 & Vinculação de conta & O usuário poderá vincular e desvincular uma conta já existente à conta do sistema. \\
\hline
RN06 & Ranking &  O ranking mostra os usuários com melhor desempenho, incentivando o aprendizado. E se o usuário ficar no top 5 do ranking, ele ganhará uma bonificação(moeda do aplicativo). \\
\hline
RN06 & Avanço imediato & Usuário pode pular de módulo se fizer um quiz de nivelamento e tiver desempenho abaixo de 90\% de perda de vida se o usuário errar mais de 3 vezes o quiz.\\
\hline
RN07 & Gratuidade & A aplicação é gratuita.\\
\hline
RN08 & Perfil & Cada usuário terá um perfil com seu nível de conhecimento (iniciante, intermediário, avançado).\\
\hline
RN09 & Progresso & O progresso do usuário será salvo automaticamente ao concluir lições ou desafios. \\
\hline
RN10 & Adaptação de ensino & O sistema adaptará os conteúdos com base no desempenho e progresso do usuário.\\
\hline
RN11 & Pontuação & A cada acerto, o usuário ganha pontos e medalhas virtuais. \\
\hline
RN12 & Pontos acumulados & Os pontos acumulados desbloqueiam novos desafios ou conteúdos bônus. \\
\hline
RN13 & Teste de nivelamento & O usuário executará um teste para medir seu conhecimento sobre educação financeira, a fim de determinar o módulo que começará e o nível pertencente. \\
\hline
RN14 & Idiomas & O usuário poderá selecionar em qual língua poderá ler o documento. \\
\hline
\end{tabular}
\end{center}


\subsection{Modelagem}
\begin{center}
 %colocar casos de uso%
\begin{figure}[H]
    \centering
    \begin{adjustbox}{max width=0.95\paperwidth,center}
        \includegraphics{graficos/bdd.png}
    \end{adjustbox}
    \fonte{bdd}
    \label{gra_logo}
\end{figure}
\end{center}




\subsection{Prototipagem}

\begin{figure}[!htbp]
    \centering
    \includegraphics[width=0.95\linewidth]{AURUM.png}
    \label{fig:enter-label}
\end{figure}





% (Aqui vanios colocaria o conteúdo da seção Prototipagem)




\subsection{Testes}

\begin{center}
\begin{tabular}{|c|p{5.5cm}|p{6.5cm}|}
\hline
\textbf{Objeto de teste} & \textbf{Resultado esperado} & \textbf{Possíveis divergências} \\
\hline
Cadastro & O cadastro deve ocorrer ao associar e-mail e senha de 6 a 16 dígitos, contendo pelo menos uma letra maiúscula, uma letra minúscula, um caractere especial e um número. & O cadastro pode ser realizado sem cumprir os requisitos da senha. \\
\hline
Tarefas & As tarefas devem ser acessadas pelo usuário de acordo com o resultado do teste de nível. & As tarefas podem não ser disponibilizadas corretamente. \\
\hline
Módulos & As tarefas devem ser atualizadas conforme o progresso do usuário. & Pode não atualizar corretamente. \\
\hline
Ranking semanal & Os usuários são classificados conforme a quantidade de pontos. & Pode não ordenar corretamente. \\
\hline
Grupos & Os usuários devem poder criar e participar de grupos. & Pode não permitir a criação ou entrada em grupos. \\
\hline
Ranking de Grupo & Classificação dos membros dos grupos com base na pontuação acumulada de cada um. & Pode não ordenar corretamente. \\
\hline
Loja & Deve oferecer itens disponíveis para compra com dinheiro do sistema acumulados. & Itens podem não aparecer ou não estar disponíveis. \\
\hline
Perfil & Permitir alteração de foto, capa e nome de usuário. & Pode não salvar ou exibir as alterações feitas. \\
\hline
Premiações & Premiações devem ser liberadas conforme desempenho do usuário. & Premiações podem não ser contabilizadas ou exibidas. \\
\hline
Teste de nível & Avalia o conhecimento do usuário para definir o módulo inicial. & Pode não direcionar corretamente o usuário para o módulo adequado. \\
\hline
Quiz & Ao ser completado com sucesso, permite avanço imediato para o próximo módulo. & Pode não liberar o módulo seguinte mesmo após o sucesso no quiz. \\
\hline
\end{tabular}
\end{center}

\section{Resultados e discussões}



\section{Considerações finais}










\postextual

% ----------------------------------------------------------
% Referências bibliográficas
% ----------------------------------------------------------

\bibliography{referencias}

% ----------------------------------------------------------
% Glossário
% ----------------------------------------------------------
%
% Há diversas soluções prontas para glossário em LaTeX. 
% Consulte o manual do abnTeX2 para obter sugestões.
%
%\glossary

% ----------------------------------------------------------
% Apêndices
% ----------------------------------------------------------

% ---
% Inicia os apêndices
% ---
\newpage
\begin{apendicesenv}

% ----------------------------------------------------------
\chapter{Pesquisa realizada}
% ----------------------------------------------------------
    A maioria das idades entre todos os participantes do formulário foi de 35 a 54 anos.\\ 
    32,7\%  dos participantes possuem Ensino Superior completo e 30,4\% Pós Graduação. \\
    53,9\%  Já tiveram contato com Educação Financeira. \\
    28,6\%  Foi apresentado a educação financeira através da internet, 22,6\% nas escolas/Instituição, 13,8\% Pela família e 29\% Nunca tiveram contato. \\
    51,6\%  sabe apenas o básico, 27,6\% tem conhecimento intermediário, 8,8\% Conhecimento  Avançado e 12\% Nenhum. \\
    53,9\% Às vezes realizam um planejamento financeiro, 26,7\% Sempre, 19,4\% nunca. \\
    37,8\% Diz poupar dinheiro todo mês, 29,5\% Não, 32,7\% Apenas quando sobra dinheiro. \\
    37,8\% Não conseguem economizar, 39,46\% Não consegue controlar os gastos, 37,3\% Não sabe como funciona investimentos, 24,9\% Não sabe organizar o orçamento mensal,  19,4\% Diz se endividar com freqüência \\
    87,1\% Querem Simulações práticas com exemplos reais e fáceis. \\
    Exposição: 
    Os dados revelam um público majoritariamente entre 35 e 54 anos, com bom nível de  escolaridade (63,1\% têm Ensino Superior completo ou Pós-graduação), mas que ainda apresenta lacunas importantes no domínio das finanças pessoais. Apesar da maioria já ter tido algum contato com educação financeira (53,9\%), mais da metade sabe apenas o básico ou nada. Isso se reflete em hábitos financeiros frágeis: apenas 26,7\% sempre fazem planejamento financeiro, e 29,5\% nunca poupam. 
\newpage
\chapter{Modelagem do Banco de Dados}
A modelagem seguiu o padrão Entidade-Relacionamento, representando as seguintes entidades principais:

Usuário: Armazena informações de acesso e pontuação do usuário.

Módulos: Agrupam aulas e desafios relacionados a temas específicos.

Aulas: Unidades de conteúdo em vídeo.

Desafios: Atividades com pontuação, associadas a módulos.

Notificações: Enviadas aos usuários com base em regras de negócio.

Relacionamentos: Foram criadas entidades associativas como “Realiza”, “Assiste” e “Desenvolve” para registrar interações entre usuários e o conteúdo.

A versão final do modelo relacional foi refinada com base nos testes do script SQL e pode ser encontrada no apêndice da documentação.
\newpage
\chapter{Páginas da prototipagem}

\begin{figure}[!htbp]
    \centering
    \includegraphics[width=0.95\linewidth]{prototipagem/2.png}
    \caption{Caption}
    \label{fig:enter-label}
\end{figure}

\begin{figure}
    \centering
    \includegraphics[width=0.95\linewidth]{prototipagem/3.png}
    \caption{Caption}
    \label{fig:enter-label}
\end{figure}

\begin{figure}
    \centering
    \includegraphics[width=0.95\linewidth]{prototipagem/4.png}
    \caption{Caption}
    \label{fig:enter-label}
\end{figure}

\begin{figure}
    \centering
    \includegraphics[width=0.95\linewidth]{prototipagem/5.png}
    \caption{Caption}
    \label{fig:enter-label}
\end{figure}

\begin{figure}
    \centering
    \includegraphics[width=0.95\linewidth]{prototipagem/6.png}
    \caption{Caption}
    \label{fig:enter-label}
\end{figure}

\begin{figure}
    \centering
    \includegraphics[width=0.95\linewidth]{prototipagem/7.png}
    \caption{Caption}
    \label{fig:enter-label}
\end{figure}

\begin{figure}
    \centering
    \includegraphics[width=0.95\linewidth]{prototipagem/8.png}
    \caption{Caption}
    \label{fig:enter-label}
\end{figure}

\begin{figure}
    \centering
    \includegraphics[width=0.95\linewidth]{prototipagem/9.png}
    \caption{Caption}
    \label{fig:enter-label}
\end{figure}

\begin{figure}
    \centering
    \includegraphics[width=0.95\linewidth]{prototipagem/10.png}
    \caption{Caption}
    \label{fig:enter-label}
\end{figure}

\begin{figure}
    \centering
    \includegraphics[width=0.95\linewidth]{prototipagem/11.png}
    \caption{Caption}
    \label{fig:enter-label}
\end{figure}

\begin{figure}
    \centering
    \includegraphics[width=0.95\linewidth]{prototipagem/12.png}
    \caption{Caption}
    \label{fig:enter-label}
\end{figure}

\begin{figure}
    \centering
    \includegraphics[width=0.95\linewidth]{prototipagem/13.png}
    \caption{Caption}
    \label{fig:enter-label}
\end{figure}

\begin{figure}
    \centering
    \includegraphics[width=0.95\linewidth]{prototipagem/14.png}
    \caption{Caption}
    \label{fig:enter-label}
\end{figure}

\begin{figure}
    \centering
    \includegraphics[width=0.95\linewidth]{prototipagem/15.png}
    \caption{Caption}
    \label{fig:enter-label}
\end{figure}

\begin{figure}
    \centering
    \includegraphics[width=0.95\linewidth]{prototipagem/16.png}
    \caption{Caption}
    \label{fig:enter-label}
\end{figure}

\begin{figure}
    \centering
    \includegraphics[width=0.95\linewidth]{prototipagem/17.png}
    \caption{Caption}
    \label{fig:enter-label}
\end{figure}

\begin{figure}
    \centering
    \includegraphics[width=0.95\linewidth]{prototipagem/18.png}
    \caption{Caption}
    \label{fig:enter-label}
\end{figure}

\begin{figure}
    \centering
    \includegraphics[width=0.95\linewidth]{prototipagem/19.png}
    \caption{Caption}
    \label{fig:enter-label}
\end{figure}



\end{apendicesenv}
% ---

% ----------------------------------------------------------
% Anexos
% ----------------------------------------------------------
\cftinserthook{toc}{AAA}
% ---
% Inicia os anexos
% ---
%\anexos
\newpage
\begin{anexosenv}

% ---
\chapter{Amostra dos gráficos produzidos pela pesquisa}
% ---
\begin{figure}[h!]
\centering
\includegraphics[width=\textwidth]{graficos/grafico1.png} % Substitua pelo nome do arquivo da imagem
\caption{gráfico etário}
\label{fig:grafico}
\end{figure}


\begin{figure}[h!]
\centering
\includegraphics[width=\textwidth]{graficos/grafico2.png}
\caption{gráfico escolaridade}
\label{fig:grafico}
\end{figure}

\begin{figure}[h!]
\centering
\includegraphics[width=\textwidth]{graficos/grafico3.png}
\caption{gráfico contato}
\label{fig:grafico}
\end{figure}

\begin{figure}[h!]
\centering
\includegraphics[width=\textwidth]{graficos/grafico4.png}
\caption{gráfico }
\label{fig:grafico}
\end{figure}

\begin{figure}[h!]
\centering
\includegraphics[width=\textwidth]{graficos/grafico5.png}
\caption{gráfico }
\label{fig:grafico}
\end{figure}

\begin{figure}[h!]
\centering
\includegraphics[width=\textwidth]{graficos/grafico6.png}
\caption{gráfico }
\label{fig:grafico}
\end{figure}

\begin{figure}[h!]
\centering
\includegraphics[width=\textwidth]{graficos/grafico7.png}
\caption{gráfico }
\label{fig:grafico}
\end{figure}

\begin{figure}[h!]
\centering
\includegraphics[width=\textwidth]{graficos/grafico8.png}
\caption{gráfico }
\label{fig:grafico}
\end{figure}

\begin{figure}[h!]
\centering
\includegraphics[width=\textwidth]{graficos/grafico9.png}
\caption{gráfico }
\label{fig:grafico}
\end{figure}

\begin{figure}[h!]
\centering
\includegraphics[width=\textwidth]{graficos/grafico10.png}
\caption{gráfico }
\label{fig:grafico}
\end{figure}


\end{anexosenv}

\end{document}
    
