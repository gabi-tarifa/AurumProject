%% abtex2-modelo-artigo.tex, v-1.9.7 laurocesar
%% Copyright 2012-2018 by abnTeX2 group at http://www.abntex.net.br/ 
%%
%% This work may be distributed and/or modified under the
%% conditions of the LaTeX Project Public License, either version 1.3
%% of this license or (at your option) any later version.
%% The latest version of this license is in
%%   http://www.latex-project.org/lppl.txt
%% and version 1.3 or later is part of all distributions of LaTeX
%% version 2005/12/01 or later.
%%
%% This work has the LPPL maintenance status `maintained'.
%% 
%% The Current Maintainer of this work is the abnTeX2 team, led
%% by Lauro César Araujo. Further information are available on 
%% http://www.abntex.net.br/
%%
% ------------------------------------------------------------------------
% ------------------------------------------------------------------------
% abnTeX2: Modelo de Artigo Acadêmico em conformidade com
% ABNT NBR 6022:2018: Informação e documentação - Artigo em publicação 
% periódica científica - Apresentação
% ------------------------------------------------------------------------
% ------------------------------------------------------------------------

\documentclass[
	% -- opções da classe memoir --
	article,			% indica que é um artigo acadêmico
	12pt,				% tamanho da fonte
	oneside,			% para impressão apenas no recto. Oposto a twoside
	a4paper,			% tamanho do papel. 
	% -- opções da classe abntex2 --
	%chapter=TITLE,		% títulos de capítulos convertidos em letras maiúsculas
	%section=TITLE,		% títulos de seções convertidos em letras maiúsculas
	%subsection=TITLE,	% títulos de subseções convertidos em letras maiúsculas
	%subsubsection=TITLE % títulos de subsubseções convertidos em letras maiúsculas
	% -- opções do pacote babel --
	english,			% idioma adicional para hifenização
	brazil,				% o último idioma é o principal do documento
	sumario=tradicional
	]{abntex2}


% ---
% PACOTES
% ---

% ---
% Pacotes fundamentais 
% ---
\usepackage{sbc-template}
\usepackage{times}			% Usa a fonte Times new Roman
\usepackage[T1]{fontenc}		% Selecao de codigos de fonte.
\usepackage[utf8]{inputenc}		% Codificacao do documento (conversão automática dos acentos)
\usepackage{indentfirst}		% Indenta o primeiro parágrafo de cada seção.
\usepackage{nomencl} 			% Lista de simbolos
\usepackage{color}				% Controle das cores
\usepackage{graphicx}			% Inclusão de gráficos
\usepackage{microtype} 			% para melhorias de justificação
\usepackage{multirow}

% ---
		
% ---
% Pacotes adicionais, usados apenas no âmbito do Modelo Canônico do abnteX2
% ---
\usepackage{lipsum}				% para geração de dummy text
% ---
		
% ---
% Pacotes de citações
% ---
\usepackage[brazilian,hyperpageref]{backref}	 % Paginas com as citações na bibl
\usepackage[alf]{abntex2cite}	% Citações padrão ABNT
% ---

% ---
% Configurações do pacote backref
% Usado sem a opção hyperpageref de backref
\renewcommand{\backrefpagesname}{Citado na(s) página(s):~}
% Texto padrão antes do número das páginas
\renewcommand{\backref}{}
% Define os textos da citação
\renewcommand*{\backrefalt}[4]{
	\ifcase #1 %
		Nenhuma citação no texto.%
	\or
		Citado na página #2.%
	\else
		%Citado #1 vezes nas páginas #2.%
        Citado nas páginas #2.%
	\fi}%
% ---

% --- Informações de dados para CAPA e FOLHA DE ROSTO ---

\title{Documento de arquitetura de software de Aurum}

\author{Francine Midori Nakasuga Oda\inst{1},Thomas Stephan Carmo\inst{1}, Arthur Silva Vieira\inst{1},\\ Gabriel Tarifa de Oliveira\inst{1}, Rafael Oliveira Muñoz\inst{1} }

\address{Instituto Federal de Educação, Ciência e Tecnologia de São Paulo - Câmpus São Paulo
}


% ---

% ---
% Configurações de aparência do PDF final

% alterando o aspecto da cor azul
\definecolor{blue}{RGB}{41,5,195}

% informações do PDF
\makeatletter
\hypersetup{
     	%pagebackref=true,
		pdftitle={\@title}, 
		pdfauthor={\@author},
    	pdfsubject={Modelo de artigo científico com abnTeX2},
	    pdfcreator={LaTeX with abnTeX2},
		pdfkeywords={abnt}{latex}{abntex}{abntex2}{atigo científico}, 
		colorlinks=true,       		% false: boxed links; true: colored links
    	linkcolor=blue,          	% color of internal links
    	citecolor=blue,        		% color of links to bibliography
    	filecolor=magenta,      		% color of file links
		urlcolor=blue,
		bookmarksdepth=4
}
\makeatother
% --- 

% ---
% compila o indice
% ---
\makeindex
% ---

% ---
% Altera as margens padrões
% ---
\setlrmarginsandblock{3cm}{3cm}{*} %{ESQUERDA}{DIREITA}
\setulmarginsandblock{3.5cm}{2.5cm}{*} %{SUPERIOR}{INFERIOR}
\checkandfixthelayout
% ---

% --- 
% Espaçamentos entre linhas e parágrafos 
% --- 

% O tamanho do parágrafo é dado por (está definido dentro do sbc-template.sty):
%\setlength{\parindent}{1.3cm}

% Controle do espaçamento entre um parágrafo e outro (está definido dentro do sbc-template.sty):
%\setlength{\parskip}{0.2cm}  % tente também \onelineskip

% Espaçamento simples
\SingleSpacing


% ----
% Início do documento
% ----
\begin{document}

% Seleciona o idioma do documento (conforme pacotes do babel)
%\selectlanguage{english}
\selectlanguage{brazil}


% Retira espaço extra obsoleto entre as frases.
\frenchspacing 

% ----------------------------------------------------------
% ELEMENTOS PRÉ-TEXTUAIS
% ----------------------------------------------------------

%---
%
% Se desejar escrever o artigo em duas colunas, descomente a linha abaixo
% e a linha com o texto ``FIM DE ARTIGO EM DUAS COLUNAS''.
% \twocolumn[    		% INICIO DE ARTIGO EM DUAS COLUNAS
%
%---

% página de titulo principal (obrigatório)
\maketitle

% titulo em outro idioma (opcional)

% inserir ABSTRACT e RESUMO aqui


% ]  				% FIM DE ARTIGO EM DUAS COLUNAS
% ---

%\begin{center}\smaller
%\textbf{Data de submissão e aprovação}: elemento c    obrigatório. Indicar dia, mês e ano

%\textbf{Identificação e disponibilidade}: elemento opcional. Pode ser indicado o endereço eletrônico, DOI, suportes e outras informações relativas ao acesso.
%\end{center}

% ----------------------------------------------------------
% ELEMENTOS TEXTUAIS
% ----------------------------------------------------------
\textual

% ----------------------------------------------------------
% Introdução
% ----------------------------------------------------------
\section{Introdução}
    %O projeto visa abranger os temas entretenimento e cultura, com seu foco em primeira instância os cinemas. https://wiki.sj.ifsc.edu.br/index.php/Dicas_de_LaTeX
    
    
A educação financeira é um direito essencial do cidadão e desempenha um papel fundamental tanto no crescimento social quanto pessoal. Contudo, apesar de sua importância, grande parte da população brasileira ainda enfrenta barreiras significativas para acessar informações adequadas sobre finanças. Isso ocorre por fatores como a falta de recursos financeiros para investir em cursos especializados, a ausência de incentivo para se aprofundar no tema desde o ensino fundamental, e até mesmo a dificuldade em compreender conceitos financeiros devido à linguagem técnica e complexa.

Segundo dados do Instituto Brasileiro de Geografia e Estatística (\citeonline{ibge2024}), embora a taxa de pobreza tenha diminuído significativamente, caindo de 67,7 milhões para 59 milhões de pessoas, ainda é preocupante o número de cidadãos que permanecem à margem da sociedade. Este cenário reforça a urgência de iniciativas que promovam a educação financeira como uma ferramenta de inclusão social e empoderamento.

Um exemplo claro de como a falta de conhecimento financeiro pode afetar a vida das pessoas é o uso do cartão de crédito. De acordo com \citeonline{pires2008planejamento}, muitos brasileiros perdem o controle dos gastos com o cartão devido à falta de educação sobre os custos reais envolvidos e a ausência de estratégias eficazes de gestão financeira. O uso excessivo do crédito e a escolha por parcelamentos com as menores parcelas possíveis frequentemente resultam em uma dívida muito maior do que o valor inicialmente contratado, agravando ainda mais a situação financeira do consumidor.

Muitos alunos da Educação para Jovens e Adultos (EJA) são trabalhadores que gerenciam diariamente suas finanças, e a educação financeira auxilia no desenvolvimento de uma cultura de prevenção e proteção, promovendo a cidadania e a melhoria da qualidade de vida, segundo \citeonline{hurtado2020importancia}. Portanto, a educação financeira desde a base deve ser prioridade.

Neste contexto, surge a necessidade de ferramentas educativas inovadoras e acessíveis. A proposta desta aplicação web é criar uma plataforma de aprendizagem interativa com foco na educação financeira. A ideia é oferecer uma experiência dinâmica e \textit{gamificada}, onde os usuários possam aprender e praticar os conceitos financeiros de forma progressiva e adaptativa, conforme seu nível de conhecimento e ritmo de aprendizado. 

Por meio de módulos interativos e desafios diários, a plataforma busca capacitar os usuários para que tomem decisões financeiras mais conscientes, como controlar gastos, entender as nuances dos investimentos, negociar dívidas e planejar para o futuro. Além disso, a utilização de tecnologia de aprendizado adaptativo permite que o conteúdo seja personalizado, atendendo às necessidades específicas de cada usuário, desde os mais iniciantes até aqueles que já possuem algum conhecimento prévio.

A missão deste projeto é expandir o acesso à educação financeira, oferecendo uma ferramenta simples, envolvente e eficaz, capaz de transformar a realidade financeira de milhares de pessoas. Ao mesmo tempo, buscamos quebrar as barreiras do entendimento, utilizando uma linguagem acessível e evitando o jargão técnico que muitas vezes afasta o público geral. O objetivo final é empoderar os cidadãos, oferecendo-lhes as ferramentas necessárias para melhorar sua qualidade de vida, reduzir as desigualdades sociais e promover um crescimento econômico mais sustentável para todos.

\subsection{Objetivos}
    Levando em consideração a necessidade de detalhes da aplicação, os objetivos foram divididos em: geral e específico.
\subsubsection{Objetivo Geral}
    O objetivo do projeto é criar uma plataforma que auxilie pessoas de todas as idades no estudo de educação financeira, fornecendo informações de forma simples para fácil entendimento, seguido de testes para treinar e avaliar seu aprendizado.
\subsubsection{Objetivos Específicos}
    Para atingir o objetivo geral do projeto foram definidos os seguintes objetivos específicos:
        \begin{enumerate}[label=\alph*)]
            \item Através de uma aplicação web, facilitar o acesso à educação financeira de maneira lúdica
            \item Incentivar o planejamento financeiro
            \item Apresentar princípios financeiros básicos 
        \end{enumerate}
       
\subsection{Justificativa}    
    A educação financeira é um elemento essencial para a construção de uma sociedade mais equilibrada e justa. Entretanto, no Brasil, muitas pessoas ainda enfrentam dificuldades para lidar com o próprio dinheiro, seja por falta de conhecimento, seja por influências do consumo excessivo. De acordo com o Instituto Brasileiro de Geografia e Estatística (IBGE), mesmo com a redução do índice de pobreza nos últimos anos, milhões de brasileiros ainda vivem em situação de vulnerabilidade econômica, muitas vezes agravada pelo endividamento descontrolado.

    Foi realizada uma pesquisa com 217 participantes, com o objetivo de compreender melhor o perfil e as dificuldades enfrentadas pelo público em relação à educação financeira. Os resultados do questionário aplicado reforçam a importância da criação de uma plataforma acessível, interativa e \textit{gamificada} sobre finanças.

    Um dos dados mais reveladores foi a resposta à pergunta "Como você descreveria seus conhecimentos em educação financeira?", em que 51,6\% dos participantes declarou possuir conhecimentos básicos e 12\% responderam que não possuem nenhum conhecimento. Isso demonstra que, embora o tema esteja ganhando mais espaço nos debates sociais, ainda há uma lacuna significativa na formação financeira da população.

    A dificuldade mais citada na pesquisa foi "tenho dificulade de controlar meus gastos", o que revela um ponto crítico: muitas pessoas desconhecem métodos de organização financeira e tomada de decisão, que é o básico. Isso limita o potencial de crescimento financeiro pessoal e mostra a urgência de métodos que desmistifiquem e tornem acessível esse tipo de conteúdo. A proposta desta plataforma contempla esse aspecto ao oferecer módulos específicos sobre controle de gastos, com explicações simples e interativas.

    Em seguida, as seguintes maiores dificuldades relatadas foram não conseguir economizar, que está diretamente ligado com a primeira maior dificuldade, e não compreender sobre como funcionam investimentos, o que reforça a necessidade de ensinar estratégias práticas de . Essa combinação entre desconhecimento técnico (com os investimentos), e dificuldade na gestão cotidiana evidencia o quanto a população precisa de uma solução completa, que ensine tanto o básico quanto o avançado.

    Outro dado relevante foi a concentração da faixa etária entre 35 e 54 anos. Esse público está em uma fase da vida onde já existe a preocupação com aposentadoria, mas muitas vezes sem ter recebido qualquer educação formal sobre finanças. Para esse grupo, uma plataforma online, moderna e compatível com os hábitos digitais representa uma alternativa prática e eficaz.

    A pesquisa também revelou que a maioria das pessoas raramente faz um planejamento financeiro, evidenciando uma falha estrutural no desenvolvimento de hábitos financeiros saudáveis. Por isso, a plataforma propõe desafios diários e objetivos de longo prazo, com o intuito de incentivar o planejamento constante e o acompanhamento do progresso do usuário.

    Quando questionados sobre o formato de aprendizado mais interessante, o público demonstrou forte preferência por formas dinâmicas e interativas, como simulações práticas e conteúdo visual. Isso valida o uso da gamificação como uma metodologia principal do projeto, com o objetivo de tornar o aprendizado mais leve, envolvente e contínuo.

    Por fim, o dado que mais reforça a relevância do projeto: a grande maioria dos participantes declarou que se interessaria por uma plataforma interativa e \textit{gamificada} para aprender sobre finanças. Isso confirma que a proposta não só responde a uma necessidade real, como também desperta o interesse do público, aumentando o engajamento e o potencial de transformação individual e social.

    Embora existam jogos que ensinem sobre educação financeira, o Aurum, além de possuir módulos interativos e desafios diários, apresenta uma abordagem inovadora que combina interatividade, \textit{gamificação} e aprendizado adaptativo. E demonstra seu diferencial por personalizar o conteúdo de acordo com as necessidades específicas de cada usuário, desde iniciantes até aqueles com conhecimento prévio, os usuários são capacitados a tomar decisões financeiras mais conscientes, como controlar gastos, compreender investimentos, negociar dívidas e planejar o futuro.
    
    Ao expandir o acesso à educação financeira, esse projeto tem o potencial de ajudar milhares de brasileiros a adquirirem mais autonomia sobre suas finanças, evitando armadilhas do consumo impulsivo e do crédito mal utilizado. Dessa maneira, além de beneficiar cada indivíduo, a iniciativa contribui para o desenvolvimento de uma sociedade mais consciente e preparada para enfrentar desafios econômicos.

    Portanto, a implementação dessa plataforma se mostra uma solução inovadora e necessária, pois busca preencher uma lacuna existente no ensino tradicional e promover um futuro financeiro mais seguro para todos.


\subsection{Requisitos Funcionais}

\begin{tabular}{|c|l|p{6cm}|}
\hline
\textbf{Código} & \textbf{Requisito} & \textbf{Descrição} \\
\hline
RF01 & Cadastrar usuário & O usuário deve ser capaz de se cadastrar na plataforma.\\
\hline
RF02 & Teste de nível & O usuário deve fazer testes para recomendar as tarefas de acordo com o nível.\\
\hline
RF03 & Disponibilização de tarefas & As tarefas devem ser disponibilizadas para o usuário de acordo com sua aptidão. \\
\hline
RF04 & Sistema de pontuação & O sistema deve contabilizar os pontos do usuário em cada tarefa realizada de acordo com o desempenho. \\
\hline
RF05 & Coleta de informações & O sistema deve coletar informações financeiras através de um quetionário para selecionar o tipo de tarefa para o usuário. \\
\hline
RF06 & Premiações por mérito & O sistema deve distribuir premiações por dias de conclusão de tarefas consecutivos, tarefas feitas com perfeição, entre outros. \\
\hline
\subsection{Requisitos Não Funcionais}

\begin{tabular}{|c|l|p{8cm}|}
\hline
\textbf{Código} & \textbf{Requisito} & \textbf{Descrição} \\
\hline
RNF01 & Disponibilidade & O servidor deve estar disponível 24 horas por dia, 7 dias por semana, com tolerância de 0,5\% de falhas. \\
\hline
RNF02 & Desempenho & O servidor deve responder em, no máximo, 0,5 segundos a todas as requisições recebidas. \\
\hline
RNF03 & Compatibilidade & Deve ser compatível com os principais sistemas operacionais (Windows, Linux). \\
\hline
RNF04 & Otimização & O tempo de resposta para atualização da contagem de pontos da aplicação deve ser de no máximo 3 segundos. \\
\hline
RNF05 & Acessibilidade & O sistema deve ser acessível para Usuários que utilizam dispositivos adaptados (Alto Contraste, Redução de Movimento e Daltonismo). \\
\hline
\end{tabular}

\hline
\hline
\end{tabular}

\subsection{Regras de Negócio}
\textbf{Código} & \textbf{Regra de Negócio} & \textbf{Descrição} \\
\hline
RN01 & Cadastro e Acesso & O usuário deve informar nome, e-mail e senha para se cadastrar. A plataforma não permitirá cadastros com e-mails já utilizados. O login só será permitido mediante autenticação por e-mail e senha válidos. \\
\hline
RM02 & Perfil e Progresso & Cada usuário terá um perfil com seu nível de conhecimento (iniciante, intermediário, avançado). O progresso do usuário será salvo automaticamente ao concluir lições ou desafios. O sistema adaptará os conteúdos com base no desempenho e progresso do usuário. \\
\hline
RN03 & Módulos de Ensino & Os módulos são liberados de forma sequencial. Cada módulo inclui teoria simples, exemplos práticos e mini-testes. O usuário só avança de nível se obtiver um desempenho mínimo de 70\% nos testes. \\
\hline
RN04 & Gamificação & A cada acerto, o usuário ganha pontos e medalhas virtuais. Os pontos acumulados desbloqueiam novos desafios ou conteúdos bônus. O ranking mostra os usuários com melhor desempenho, incentivando o aprendizado. \\
\hline
RN05 & Notificações e Lembretes & O sistema envia lembretes diários para os usuários que ativarem essa opção. Notificações são usadas para informar novos conteúdos ou desafios disponíveis. \\
\hline
RN06 & Planejamento Financeiro & A plataforma oferece uma ferramenta básica de planejamento financeiro. O usuário pode registrar receitas, despesas e metas mensais. O sistema envia alertas se os gastos superarem o limite definido. \\
\hline
RN07 & Feedback e Suporte & O usuário poderá avaliar cada módulo após sua conclusão. Um canal de ajuda estará disponível para dúvidas ou problemas técnicos. \\
\hline
\end{tabular}


\documentclass{.}





\usepackage{.} % Este pacote é usado para criar tabelas com linhas mais suaves

\postextual

% ----------------------------------------------------------
% Referências bibliográficas
% ----------------------------------------------------------

\bibliography{referencias}

% ----------------------------------------------------------
% Glossário
% ----------------------------------------------------------
%
% Há diversas soluções prontas para glossário em LaTeX. 
% Consulte o manual do abnTeX2 para obter sugestões.
%
%\glossary

% ----------------------------------------------------------
% Apêndices
% ----------------------------------------------------------

% ---
% Inicia os apêndices
% ---
\newpage
\begin{apendicesenv}

% ----------------------------------------------------------
\chapter{Pesquisa realizada}
% ----------------------------------------------------------
    A maioria das idades entre todos os participantes do formulário foi de 35 a 54 anos.\\ 
    32,7\%  dos participantes possuem Ensino Superior completo e 30,4\% Pós Graduação. \\
    53,9\%  Já tiveram contato com Educação Financeira. \\
    28,6\%  Foi apresentado a educação financeira através da internet, 22,6\% nas escolas/Instituição, 13,8\% Pela família e 29\% Nunca tiveram contato. \\
    51,6\%  sabe apenas o básico, 27,6\% tem conhecimento intermediário, 8,8\% Conhecimento  Avançado e 12\% Nenhum. \\
    53,9\% Às vezes realizam um planejamento financeiro, 26,7\% Sempre, 19,4\% nunca. \\
    37,8\% Diz poupar dinheiro todo mês, 29,5\% Não, 32,7\% Apenas quando sobra dinheiro. \\
    37,8\% Não conseguem economizar, 39,46\% Não consegue controlar os gastos, 37,3\% Não sabe como funciona investimentos, 24,9\% Não sabe organizar o orçamento mensal,  19,4\% Diz se endividar com freqüência \\
    87,1\% Querem Simulações práticas com exemplos reais e fáceis. \\
    Exposição: 
    Os dados revelam um público majoritariamente entre 35 e 54 anos, com bom nível de  escolaridade (63,1\% têm Ensino Superior completo ou Pós-graduação), mas que ainda apresenta lacunas importantes no domínio das finanças pessoais. Apesar da maioria já ter tido algum contato com educação financeira (53,9\%), mais da metade sabe apenas o básico ou nada. Isso se reflete em hábitos financeiros frágeis: apenas 26,7\% sempre fazem planejamento financeiro, e 29,5\% nunca poupam. 

\end{apendicesenv}
% ---

% ----------------------------------------------------------
% Anexos
% ----------------------------------------------------------
\cftinserthook{toc}{AAA}
% ---
% Inicia os anexos
% ---
%\anexos
\newpage
\begin{anexosenv}

% ---
\chapter{Amostra dos gráficos produzidos pela pesquisa}
% ---
\begin{figure}[h!]
\centering
\includegraphics[width=\textwidth]{graficos/grafico1.png} % Substitua pelo nome do arquivo da imagem
\caption{gráfico etário}
\label{fig:grafico}
\end{figure}


\begin{figure}[h!]
\centering
\includegraphics[width=\textwidth]{graficos/grafico2.png}
\caption{gráfico escolaridade}
\label{fig:grafico}
\end{figure}

\begin{figure}[h!]
\centering
\includegraphics[width=\textwidth]{graficos/grafico3.png}
\caption{gráfico contato}
\label{fig:grafico}
\end{figure}

\begin{figure}[h!]
\centering
\includegraphics[width=\textwidth]{graficos/grafico4.png}
\caption{gráfico }
\label{fig:grafico}
\end{figure}

\begin{figure}[h!]
\centering
\includegraphics[width=\textwidth]{graficos/grafico5.png}
\caption{gráfico }
\label{fig:grafico}
\end{figure}

\begin{figure}[h!]
\centering
\includegraphics[width=\textwidth]{graficos/grafico6.png}
\caption{gráfico }
\label{fig:grafico}
\end{figure}

\begin{figure}[h!]
\centering
\includegraphics[width=\textwidth]{graficos/grafico7.png}
\caption{gráfico }
\label{fig:grafico}
\end{figure}

\begin{figure}[h!]
\centering
\includegraphics[width=\textwidth]{graficos/grafico8.png}
\caption{gráfico }
\label{fig:grafico}
\end{figure}

\begin{figure}[h!]
\centering
\includegraphics[width=\textwidth]{graficos/grafico9.png}
\caption{gráfico }
\label{fig:grafico}
\end{figure}

\begin{figure}[h!]
\centering
\includegraphics[width=\textwidth]{graficos/grafico10.png}
\caption{gráfico }
\label{fig:grafico}
\end{figure}


\end{anexosenv}

\end{document}
    
